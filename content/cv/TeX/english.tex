%\title{My two column CV}
%
% tccv (two columns curriculum vitae) is a LaTeX class inspired by
% the template found at latextemplates.com by Alessandro Plasmati.
%
% Create by Nicola Fontana, the original files can be downloaded from:
% http://dev.entidi.com/p/tccv/
%
\documentclass{tccv}
\usepackage[english,brazil]{babel}
\usepackage{xcolor}

\begin{document}

\part{Victor Carreira}

\section{Previous Experience}

\begin{eventlist}



\item{March 2013 -- March 2014}
     {National Observatory}
     {Field Surveying}


Worked in the field survey for acquiring gravimetric, magnetic, magnetotelluric data with complementary differential GPS inside Parnaíba Sedimentary Basin area. Official contract between the National Observatory (ON) and the company British Petroleum (BP).



\item{May 2011 -- May 2012}
     {National Observatory}
     {Cnpq Scientific Research - Level I}
    
Magnetic susceptibility characterization  of rocks for the interpretation of magnetometric surveys.



\item{November 2011 -- November 2012}
     {Federal University Of Rio de Janeiro}
     {Petrobras Trainer}

Biostratigraphic correlation in deep waters of the South Atlantic Brazilian sedimentary basins, among the Cretaceous Period, through the use of radiolarian biostratigraphy.



\item{February 2006 -- May 2007}
      {Tecma}
      {Chemical Technician}

Microbiological and chemical analysis in water and soil samples for heavy metals using atomic absorption technique, and volatile semi-volatile organic compounds (BTEX, VOC, SVOCs)  analysis made by gas and liquid chromatography technique.

\end{eventlist}

\section{Educational Information}

\begin{yearlist}
	
	\item{2017-today} 
	      {PhD in Geophysics}
	      {National Observatory, ON}

\item{2013- 2015}
     {Masters Degree in Geophysics}
     {National Observatory, ON}

\item{2007 -- 2012}
     {Geology Bachelor Degree}
     {Federal University of Rio de Janeiro, UFRJ}

\item{2002 -- 2004}
     {Chemical Technician}
     {Federal Center of Technological Education of Chemistry, CEFETEQ}

\item{2001 -- 2003}
     {High School}
     {Federal Center of Technological Education of Chemistry, CEFETEQ}

\end{yearlist}
\personal
    [https://github.com/VictorCarreira]
    {Brazil, Rio de Janeiro, Rio de Janeiro, Manoel Magiolli Street 37. ZIP Code: 21940-270}
    {(55+21) 998288484}
    {carreiravr@gmail.com}
    \\ \\
\colorbox{green}{
\begin{minipage}{0.95\linewidth}
    Professional Registers:\\
    CREA: 2012131104\\
    CRQ: 03422220
\end{minipage}
}

\section{Extra Curricular Activities}

\begin{yearlist}

\item{2018}
     {80th EAGE Annual International Conference and Exhibition, Copenhagen, Denmark.}
     {A Comparison of Machine Learning Processes for Classification of Rock Units Using Well Log Data.}

\item{2018}  
     {Brazilian Congress of geology}
     {Short course: Gravimetric model using pontual sources.}
     
\item{2016}
	 {Geophysical National Symposium}
	 {Conference paper "Basement Relief Recovery for Parana Sedimentary Basin trough creating a Resistivity Composite 1D section model for the Central Region of the Parana Basin, South Central Brazil." the $ 7 ^ {o} $ National Symposium of the Brazilian Society of Geophysics, from 25 to October 27, 2016.}

\item{2015}
	 {Geophysical International Congress}
	 {Congress Class "Model for the Limit  of crust-mantle through Gravimetry aerial use in central Region the Parana Sedimentary Basin, southern part of Brazil," the $ 14 ^ {o} $ International Congress of the Brazilian Society of Geophysics, held from 3 to August 6, 2015.}
	 

%\item{2015}
%	 {Mesa Redonda}
%	 {Participou da mesa redonda "O Papel do Ge\'ologo no S\'eculo XXI", no dia 10 de abril de 2015, durante a I Jornada Xisto do Calouro, evento realizado na UFRJ.}

\item{2014}
	 {Brazilian Congress of Geology}
	 {Congress Poster "Revision of a structural composite cratonic unities and mobile belts basement of the Parana Sedimentary Basin through the use of potential methods Imaging "at $ 47 ^ {o} $ Brazilian Geology Congress, held from 21 to 26 September 2014.}

%\item{2013}
%     {Curso}
%     {Participou do curso "Geof\'isica B\'asica" promovido pela empresa $Gesoft^{\textregistered}$, nos dias 29, 30, 31 de outubro de 2013, com duração de 25 horas.}

\item{2013}
     {Curso}
     {Course "Theoretical and practical for the fundamentals of magnetotelluric, gravimetric and magnetic methods," held at the National Observatory from 3 to 7 June 2013. }

%\item{2012}
%     {Congresso Brasileiro de Geologia}
%     {Apresentou o trabalho “A hist\'oria da Geof\'isica”, no simp\'osio tem\'atico
%filosofia das geoci\^encias do $46^{o}$ congresso brasileiro de geologia, realizado de 30 de setembro a 5 de outubro de 2012.}

%\item{2011}
%     {Jornada de Inicia\c{c}\~ao Cinet\'ifica}
%     {Apresentou o trabalho de inicia\c{c}\~ao cinet\'ifica “An\'alise das medidas de suscetibilidade
%magn\'etica de rochas para a interpreta\c{c}\~ao de levantamentos magnetom\'etricos das regi\~oes
%de Cabo frio e da Ilha de trindade, Rio de Janeiro, Brasil.”, no ano de 2011 no Observat\'orio
%nacional}

%\item{2010}
%     {Congresso Brasileiro de Geologia}
%     {Apresentou o trabralho "Investiga\c{c}\~ao sobre a influ\^encia dos eventos clim\'aticos globais do pale\'ogeno em associa\c{c}\~oes de radiol\'arios na bacia de santos, margem continental leste brasileira.", no $45^{o}$ Congresso Brasileiro no ano de 2010}

%\item{2010}
%     {Jornada de Inicia\c{c}\~ao Cinet\'ifica}
%     {Apresentou o trabalho de inicia\c{c}\~ao cient\'ifica “An\'alise da distribui\c{c}\~ao estratigr\'afica e
%prefer\^encias termais das associa\c{c}\~oes de radiol\'arios em uma se\c{c}\~ao do pale\'ogeno, na
%bacia de santos, margem continental leste brasileira.”, no ano de 2010}
%
%\item{2010}
%     {Jornada de Inicia\c{c}\~ao Cinet\'ifica}
%     {Apresentou o trabalho de Inicia\c{c}\~ao cient\'ifica “An\'alise modal com o uso do Adobe
%Photoshop cs4, no ano de 2010.}
\end{yearlist}

\section{Idioms}

\begin{factlist}
\item{English}{Fluent}
\item{Spanish}{Intermediate}
\item{Russian}{Beginner}
\end{factlist}

%\section{Realiza\c{c}\~oes}

%\begin{yearlist}

%\item{2015}
  %   {T\'itulo de Mestre em Geof\'isica}

%\item{2012}
 %    {T\'itulo de Bacharel em Geologia}

%\item{2010}
 %    {Funda\c{c}\~ao da Empresa Xisto Jr.}

%\end{yearlist}

\section{Computational Skills}

\begin{factlist}

\item{Advanced Level}
	 {Python, FORTRAN, GitHub, \TeX, ArcGis$^{\textregistered}$, Quantum Gis, Oasis Montaj$^{\textregistered}$, Target$^{\textregistered}$, GMSYS$^{\textregistered}$, Winglink$^{\textregistered}$}
\item{Intermediate Level}
     {MarkDown, Octave, Matlab, R, GitHub}
\item{Basic Level}
     {CCS, HTML, PhP}

\end{factlist}

\end{document}
